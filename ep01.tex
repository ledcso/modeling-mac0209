\documentclass{article}

% Language setting
% Replace `english' with e.g. `spanish' to change the document language
%\usepackage[portuguese]{babel}

% Set page size and margins
% Replace `letterpaper' with `a4paper' for UK/EU standard size
\usepackage[letterpaper,top=2cm,bottom=2cm,left=3cm,right=3cm,marginparwidth=1.75cm]{geometry}

% Useful packages
\usepackage{amsmath}
\usepackage{graphicx}
\usepackage[colorlinks=true, allcolors=blue]{hyperref}

\title{Relatório do EP de MAC0209}
\author{Andre Neves, Luis Vergar, Thiago}

\begin{document}
\maketitle


\begin{abstract}
In digital signal processing, the Nyquist-Shannon sampling theorem establishes that the sample rate must be at least twice the bandwidth of the signal to avoid aliasing. 
We study the functions sine, logarithm, and arctan, do a sampling of them, and quantize the data to reconstruct the signal into a discrete-time one. 

\end{abstract}

\newpage

\tableofcontents

\newpage

\section{Introduction}

In digital electronics, digital signals are represented as pulse waves, which are usually generated by the switching of a transistor. 
Due to the nature of pulse waves, signal processing is best done through the use of discrete-time functions, in contrast to raw signals which are usually represented byu continuous-time functions. 
In that vein, we convert the continuous-time signal to a discrete-time one, through a process called quantization. 
The first step in the quantization of a signal is sampling, which involves getting a "sample" of the original continuous-time function and use it to reconstruct the signal to obtain a discrete-time one. 

\section{Objectives}

To understand the concepts of precision, accuracy, and significant figures, as well as analyze the errors from sampling, taking into account the Nyquist-Shannon theorem. 

\section{Schedule}

Schedule:  

2025-03-06: Analyze the problem statement.  

2025-03-08: Write the code for the plots.  

2025-03-10: Write the code for plotting the errors.  

2025-03-26: Analyse the results and write the paper.  

\section{Dados e métodos}

Explique os dados usados e os métodos desenvolvidos.

\section{Resultados experimentais}

Apresente os resultados obtidos, Explore tabelas e gráficos ilustrativos. Interprete os resultados e apresente uma visão crítica.

\section{Discussion}

We applied the Nyquist-Shannon theorem when choosing our frequency for sampling to avoid aliasing during the interpolation of the discrete-time signal, and verified the theorem by getting a new function with a frequency not lower than the original continuous-time signal. 
Even though the three functions were simple ones, the functions sine, log, and arctan, are representative of real-life signals, and let us apply quantization and the Nyquist-Shannon theorem in a controlled manner.  

\end{document}
