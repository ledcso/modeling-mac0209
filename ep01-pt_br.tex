\documentclass{article}

% Language setting
% Replace `english' with e.g. `spanish' to change the document language
%\usepackage[portuguese]{babel}

% Set page size and margins
% Replace `letterpaper' with `a4paper' for UK/EU standard size
\usepackage[letterpaper,top=2cm,bottom=2cm,left=3cm,right=3cm,marginparwidth=1.75cm]{geometry}

% Useful packages
\usepackage{amsmath}
\usepackage{graphicx}
\usepackage[colorlinks=true, allcolors=blue]{hyperref}

\title{Relatório do EP de MAC0209}
\author{Andre Neves, Luis Vergara, Thiago}

\begin{document}
\maketitle


\begin{abstract}
No processamento de sinais digitais, o teorema de amostragem de Nyquist-Shannon estabelece que a taxa de amostragem deve ser pelo menos duas vezes a largura de banda (bandwidth) do sinal para evitar aliasing.
Estudamos as funções seno, logaritmo e arctan, fazemos uma amostragem delas e quantizamos os dados para reconstruir o sinal em um sinal de tempo discreto.

\end{abstract}

\newpage

\tableofcontents

\newpage

\section{Introdução}

Na eletrônica digital, os sinais digitais são representados como ondas de pulso, que geralmente são geradas pela comutação (switching) de um transistor.
Devido à natureza das ondas de pulso, o processamento de sinal é melhor feito por meio do uso de funções de tempo discreto, em contraste com sinais brutos que geralmente são representados por funções de tempo contínuo.
Nesse sentido, convertemos o sinal de tempo contínuo em um de tempo discreto, por meio de um processo chamado quantização.
O primeiro passo na quantização de um sinal é a amostragem, que envolve obter uma "amostra" da função de tempo contínuo original e usá-la para reconstruir o sinal para obter uma de tempo discreto.

\section{Objetivos}

Entender os conceitos de precisão, acuracia e algarismos significativos, bem como analisar os erros da amostragem, levando em consideração o teorema de Nyquist-Shannon.

\section{Cronograma}

Cronograma:  

2025-03-06: Analisar o enunciado do problema.  

2025-03-08: Escrever o codigo para plotar o grafico.  

2025-03-10: Plotar o erro absoluto.  

2025-03-26: Analisar resultados e escrever o relatório.  

\section{Dados e métodos}

\section{Resultados}

O periodo da função seno é $2\pi$. A frequência de amostragem de nosso experimento foi:  
$$\displaystyle\frac{10}{10-(-10)} = 0.5 \text{ Hertz (amostras por segundo)}$$

\section{Discussão}

Aplicamos o teorema de Nyquist-Shannon ao escolher nossa frequência para amostragem para evitar aliasing durante a interpolação do sinal de tempo discreto, e verificamos o teorema obtendo uma nova função com uma frequência não menor que o sinal de tempo contínuo original.
Embora as três funções fossem simples, as funções seno, log e arctan são representativas de sinais da vida real, e nos permitem aplicar a quantização e o teorema de Nyquist-Shannon de forma controlada.

\end{document}
